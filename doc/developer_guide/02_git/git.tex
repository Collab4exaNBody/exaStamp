\section{Using git}
\label{sec:install}

\begin{verbatim}

Setup

Prerequisites
install necessary packages as in scripts/required-packages-Ubuntu-18.04.sh

1. clone your own git repository
git clone ssh://your-inti-login@inti.ocre.cea.fr:/ccc/home/cont001/xstampdev/xstampdev/ExaStamp.git exaStamp

2. première configuration et compilation
<path-to-sources>/scripts/configure-Ubuntu-18.04.sh


How to add a remote connection to inti if you cloned from another location

3. 
git remote add inti ssh://votre-login-inti@inti.ocre.cea.fr:/ccc/home/cont001/xstampdev/xstampdev/ExaStamp.git

3b. pour faire un push ou un pull depuis/vers un depot distant :
git push inti ma-branche # push vers inti
git pull inti ma-branche # pull depuis inti

Note:
vous aurez toujours un 'remote' nommé origin qui est l'URL que vous avez utilisez pour le clone
donc si vous voulez push/pull vers le serveur duquel vous avez cloné :
git push origin ma-branche
git pull origin ma-branche

Note 2:
pour voir a liste des depot remote
git remote -v


Handling branches
copy a remote branch to the local repo, with exactly the same state as the remote one :
git checkout -b <my_new_branch> <remote>/<branch_name>
or
git reset --hard <remote>/<branch_name>

\end{verbatim}


